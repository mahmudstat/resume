\documentclass{article}
\usepackage[
  letterpaper,
  margin=1in,
  headsep=4pt, % separation between header rule and text
]{geometry}
\usepackage{xcolor}
\usepackage[colorlinks = true,
            linkcolor = blue,
            urlcolor  = blue,
            citecolor = blue,
            anchorcolor = blue]{hyperref}
\usepackage{fancyhdr}
\usepackage{tgschola}
\usepackage{lastpage}

\pagestyle{fancy}
\fancyhf{}
\fancyhead[C]{%
  \footnotesize\sffamily
  \yourname\quad
  web: \textcolor{blue}{\itshape\yourweb}\quad
  \textcolor{blue}{\youremail}}
\fancyfoot[C]{Page \thepage\ of \pageref{LastPage}}

\newcommand{\soptitle}{Statement of Purpose}

\newcommand{\yourname}{Abdullah Al Mahmud}
\newcommand{\youremail}{almahmud.sbi@gmail.com}
\newcommand{\yourweb}{https://mahmud.statmania.info}

\newcommand{\statement}[1]{\par\medskip
  \underline{\textcolor{blue}{\textbf{#1:}}}\space
}


\begin{document}

\begin{center}\LARGE\soptitle\\
\large of \yourname\ (Mathematics (Statistics) PhD applicant for Fall---2021)
\end{center}

\hrule
\vspace{1pt}
\hrule height 1pt

\bigskip

In this statement, I present how I became enthusiastic for advanced statistical research and my motivation for applying to the PhD program at \textbf{University of Louisiana at Lafayette}. \\

From my childhood, deeply engrained in my heart was an instinctive desire to become a researcher. With the passage of time, however, I became somewhat indifferent to my ambition, choosing statistics as my undergraduate major only reluctantly, without then knowing the discipline is one of the most perfect trajectories to my inborn goal, thus securing poor grades in the first year. However, when I became a sophomore, I started to have feelings for statistics. The ability of probability distributions and statistical inferential techniques to model and predict real-world scenarios captivated me. That is when I determined my future career path: building a solid foundation of statistical theories and applying them to the field of cosmology --- the discipline I most liked and explored informally --- in an effort to work to uncover the mysteries of the universe. \\

Out of my exploration into cosmology, I had been prolifically contributing articles to national science magazine. Overtime, however, I controlled my enthusiasm for writing, concentrating more on learning statistical theory and applications. Almost all statistical concepts such sampling methods; sampling distributions; inferential techniques such as maximum likelihood estimator and Bayes estimator; Rao-Blackwell theorem; Lehmann-Scheff\'e theorem; Box-Muller algorithm; logit model; various test statistics; time series analysis; and stochastic processes--- all grabbed my attention. Now enthralled by statistics, my grades also improved. When in 3rd year, I started reading scientific papers on statistics, mostly with historical importance. Then, for several months, I explored quasi-probability distributions and its possible applications. Then I started to work with Benford's law, a law stating that most numbers start with smaller digits, and I presented a conference paper on the relationship of the law with the uniform distribution. In another work, which is going to be published on a national journal , I demonstrated that, like many other real-world data sets, disaster death tolls conform to the law well. In the meantime, thanks to my increased commitment to statistics and responsiveness in classes, I got the chance to lead a research group of 11 members in fourth year. We conducted research on 'An Insight into Induced Seismicity in Bangladesh', and found significant finding, which is now being prepared for \href{https://drive.google.com/open?id=1PSrhBG65dMmJDyyjU44Kg5Ub5rmmIlyt}{publication}. 
In parallel, I founded and have been writing on Stat Mania, a portal dealing with statistical concepts and their implementation through, chiefly, R programming language. Thanks to these activities and attaining familiarity as a writer, I joined 'Statistical Research Consultants, Bangladesh' as a content developer.After finishing my B.Sc. and M.Sc. degrees in statistics from the University of Dhaka, I joined EAL as a research assistant, and then switched to Pabna Cadet College as lecturer in statistics. These employments made my desire to pursue higher study in sttaistics and research stronger. \\

Toward the end of undergraduate courses, added to my curiosity was machine learning. I took some MOOC courses and went through several books, being especially benefited by the book \textit{Statistical Learning} by Trevor Hastie and Rob Tibshirani. It is known that the parametric algorithms, when the underlying assumptions are met, are faster and allow for simpler interpretations. Thus, I intend to work to improve and employ parametric machine learning algorithms with the help of econometric and advanced regression techniques, and to use the developments along with along with existing statistical methods for astronomical research.  \\

History is abundant with instances of mathematicians and statisticians contributing to the development of astronomy and cosmology. Carl Freidrich Gauss, who discovered the normal distribution and contributed to the method of least squares, had significant role in orbital predication of celestial bodies. The discovery concerning expanding universe was accomplished by means of \href{https://www.pnas.org/content/pnas/15/3/168.full.pdf}{bivariate analysis of distances and redshifts of galaxies}. Also, one of the most active researches in astronomy involves a mysterious star, called Tabby's star, which is being investigated through statistical data analysis. Due to the ability of statistics to play a huge role in astronomy, there have been some recent initiatives to formalize the association between the two disciplines. This is, perhaps, best epitomized by the founding of International Astrostatistics Association (IAA) in 2012. \\

To remain in touch with astronomy for the purpose of prospective future association with the discipline, I founded and have been occasionally writing on the portal Bishwo.com. I also wrote for magazines such as \href{http://z2i.org}{Zero to Infinity}, \href{http://www.byapon.com}{Byapon}, \href{https://www.facebook.com/bigganchinta/}{Bigganchinta} etc., a practice which culminated in writing \href{https://mahmud.statmania.info/r\%C3\%A9sum\%C3\%A9.html#science-books}{several science books}, one of them being the translation of \textit{A Briefer History of Time} by Dr. Stephen Hawking and Leonard Mlodino. Also, I continuously learn and apply, both out of necessity and curiosity, many different programming languages, such as R, Bash, js etc.\\

Perhaps, the greatest desire of the present physicists and cosmologists is to discover a universal theory, called the theory of everything, which can merge all current physical laws describing the universe into one single theory. Statistics can propel the task, one possible approach being employing robust parametric machine learning algorithms and testing which model best fits the observations. I long for being a part such a research which has the potential to make arguably the biggest scientific discovery in the history of mankind. \\

I am truly fascinated at the courses and their contents offered by University of Louisiana at Lafayette, and I strongly believe they will prepare me to engage in astrostatistical research. While I feel that it is huge commitment to pursue graduate study, I love and am willing to take challenge and work under pressure to pursue my dream. Hence, I appeal to the graduate admission committee to consider me for admission along with financial assistance. 

%\statement{Project \#1}



\end{document}